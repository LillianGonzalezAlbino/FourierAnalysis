\documentclass{article}
\usepackage[utf8]{inputenc}
\usepackage[english]{babel}
\usepackage{amsthm}
\usepackage{blindtext}
\usepackage{amssymb}
\usepackage[options]{algorithm2e}

%\newtheorem{theorem}{Theorem}
\newtheorem{theorem}{Theorem}[section]
\newtheorem{corollary}{Corollary}[theorem]
\newtheorem{lemma}[theorem]{Lemma}

\theoremstyle{remark}
\newtheorem*{remark}{Remark}

\theoremstyle{definition}
\newtheorem{definition}{Definition}[section]

\title{Fourier Analysis and Deep Learning Technical Report}
\author{Rosa Garza & Lillian González Albino & Sylvia Nwakanma}
\date{July 2018}

\begin{document}

\maketitle

\begin{abstract}
    In this research, we apply Fourier analysis as a tool to extract important features from a given dataset and use the features as predictors for a deep learning model. As opposed to other methods like the spacial model, the Fourier analysis provides precise information about the behavior of coalitions while not removing any of the data. Using spectral analysis it is easy to interpret and find the coalitions with the largest effects on the data thus making them good predictors. Since we will be working with large datasets, we will use a partitioning technique described in \cite{David} that will reduce the size or our working set by  removing redundant information without altering the data.  
\end{abstract}

\section{Preliminaries}
    \begin{definition}\cite{Jgraph} The \textbf{Johnson Graph} $J(n,k)$ has vertices given by the k-subsets of $\{1,\cdots,n\}$ with two vertices connected if and only if their intersection has size $k-1$.
    \end{definition}

\section{Work}
    Algorithm to compute adjacency matrix from a Johnson graph. \\
    \begin{algorithm}[H]
        \SetAlgoLined
        \KwResult{Eigen vectors of Adjacency Matrix of Johnson Graph}
        initialize\;
        Set $alphabet$ of size $n$ in lexicographical order\;
        Set $k$ length of pairs\;
        Define $tuple list$ = combinations of $alphabet$ of length $k$, no repetition, in lexicographical order\;
        Define $Adj$ = zero matrix of size $\binom{n}{k}*\binom{n}{k}$\;
        \For{$i$ in $tuple list$ \textbf{as} row index}
        {
            \For{$j$ in $tuple list$ \textbf{as} column index}
            {
                \If{$|tuple list\lbrack i\rbrack  \cap tuple list\lbrack j\rbrack|$ = $k-1$}{change $Adj\lbrack i,j\rbrack = 1$}
            }
        }
        Define $eigen vecs$ = eigen vectors of $Adj$\;
    \end{algorithm}
\section{Future Work}


\begin{thebibliography}{}
    \bibitem{David}Uminsky, David Thomas. “Generalized Spectral Analysis for Large Sets of Approval Voting Data.” Harved Mudd College, 2003, pp. 19–33.

    \bibitem{Court} Lawson, B., Orrison, M., & Uminsky, D. (2006). Spectral analysis of the supreme court. Mathematics Magazine, 79(5), 340-346. h p://dx.doi.org/10.2307/27642969
    
    \bibitem{Jgraph} Weisstein, Eric W. "Johnson Graph." From MathWorld--A Wolfram Web Resource. http://mathworld.wolfram.com/JohnsonGraph.html

\end{thebibliography}

\end{document}
